\documentclass[11pt,a4paper]{article}
\usepackage[margin=1in]{geometry}
\usepackage[T1]{fontenc}
\usepackage[utf8]{inputenc}
\usepackage{xcolor}
\usepackage{listings}
\usepackage{titlesec}
\usepackage{hyperref}

\definecolor{codebg}{RGB}{248,248,248}
\definecolor{keyword}{RGB}{0,0,180}
\definecolor{comment}{RGB}{0,128,0}
\definecolor{string}{RGB}{163,21,21}

\lstdefinestyle{codestyle}{
  backgroundcolor=\color{codebg},
  basicstyle=\ttfamily\small,
  keywordstyle=\color{keyword}\bfseries,
  commentstyle=\color{comment}\itshape,
  stringstyle=\color{string},
  numberstyle=\tiny\color{gray},
  numbers=left,
  stepnumber=1,
  numbersep=8pt,
  tabsize=2,
  showstringspaces=false,
  breaklines=true,
  frame=single,
  framerule=0.25pt,
  rulecolor=\color{gray},
  captionpos=b
}

\titleformat{\section}{\Large\bfseries}{}{0pt}{}
\titleformat{\subsection}{\normalsize\bfseries}{}{0pt}{}

\begin{document}
\begin{center}
  {\LARGE Assignment 7 — All Source Code}\\[4pt]
  {\small Exported into a single PDF}
\end{center}
\bigskip

\section*{Root}

\subsection*{main.py}
\begin{lstlisting}[style=codestyle,language=Python]
# This line imports the 'os' module, which contains functions for working with the operating system
# (like creating folders, files, etc.)
import os

def create_task_folders():
    """
    This is a function definition. Functions are like mini-programs that do a specific task.
    This function will create 4 folders named TASK1, TASK2, TASK3, and TASK4.
    The text between triple quotes is called a "docstring" - it explains what the function does.
    """
    
    # This is a for loop. It will repeat the code inside it multiple times.
    # range(1, 5) creates a sequence: 1, 2, 3, 4 (note: it stops before 5)
    # So 'i' will take the values 1, 2, 3, and 4 in each iteration
    for i in range(1, 5):
        # f"TASK{i}" is called an f-string (formatted string). 
        # It creates a string where {i} gets replaced with the current value of i
        # So when i=1, folder_name becomes "TASK1"
        # When i=2, folder_name becomes "TASK2", and so on
        folder_name = f"TASK{i}"
        
        # try-except is used for error handling
        # If something goes wrong in the 'try' block, the code jumps to 'except'
        try:
            # os.makedirs() is a function that creates a folder (directory)
            # folder_name is the name of the folder we want to create
            # exist_ok=True means "don't show an error if the folder already exists"
            os.makedirs(folder_name, exist_ok=True)
            
            # print() displays text on the screen
            # This will show which folder was created successfully
            print(f"Created folder: {folder_name}")
            
        except Exception as e:
            # This runs only if there was an error in the try block
            # 'e' contains information about what went wrong
            print(f"Error creating folder {folder_name}: {e}")

# This is a special Python condition that checks if this file is being run directly
# (not imported from another file). It's a common Python pattern.
if __name__ == "__main__":
    # These print statements will show when the program starts and finishes
    print("Creating task folders...")
    
    # This calls our function to actually create the folders
    create_task_folders()
    
    print("Task folders creation completed!")
\end{lstlisting}

\subsection*{marge.py}
\begin{lstlisting}[style=codestyle,language=Python]
#!/usr/bin/env python3
import os

def main():
    # Ask for assignment number
    assignment_num = input("Enter assignment number (e.g., 7): ")
    
    root = os.path.abspath(os.path.dirname(__file__))

    # Look for TASK folders (uppercase)
    tasks = [
        d for d in os.listdir(root)
        if os.path.isdir(os.path.join(root, d)) and d.startswith("TASK")
    ]
    tasks.sort(key=lambda d: int(d.replace("TASK", "")) if d.replace("TASK", "").isdigit() else d)

    # Create output filename with assignment number
    out_path = os.path.join(
        root,
        f"Assignment{assignment_num.zfill(2)}_1000054254_MdMonjurulHasanBhuiyan.txt"
    )

    with open(out_path, "w", encoding="utf-8") as out:
        for task in tasks:
            num = task.replace("TASK", "")
            # Write folder name in lowercase
            out.write(f"//task{num}\n")

            task_dir = os.path.join(root, task)
            # Get all Java files (excluding tester files)
            java_files = sorted(
                f for f in os.listdir(task_dir)
                if f.endswith(".java") and "Tester" not in f
            )

            for fname in java_files:
                file_path = os.path.join(task_dir, fname)
                with open(file_path, "r", encoding="utf-8") as f:
                    content = f.read()
                # Only include files that DON'T have main method (design files)
                if "public static void main(String[] args)" not in content:
                    out.write(content)
                    out.write("\n")

            out.write("\n")

    print(f"✔️  Written design classes to {out_path}")

if __name__ == "__main__":
    main()
\end{lstlisting}

\clearpage
\section*{TASK1}

\subsection*{Player .java}
\begin{lstlisting}[style=codestyle,language=Java]
class Player{
    static int total = 0;
    static String arrname[] = new String [11];

    public String name;
    public String country;
    public int rating;

    Player(String name, String country, int rating){
        this.name = name;
        this.country = country;
        this.rating = rating;
        total++;
        arrname[total] = name;
    }

    public String player_detail(){
        return "Player Name: " + name + "\nCountry: " + country + "\nJersey Number: " + rating;
    }
    public static void info(){
        System.out.println("Total number of players: " + total);
        System.out.println("Players enlisted so far:");
        for(int i = 0; i < total; i++){
            if(arrname[i] != null){
                System.out.print(arrname[i] + " , ");
            }
        }
        System.out.println();
    }

}
\end{lstlisting}

\subsection*{PlayerTester.java}
\begin{lstlisting}[style=codestyle,language=Java]
public class PlayerTester{
    public static void main(String[] args) {
      System.out.println("Total number of players: " + Player.total);
      System.out.println("1------------------");
      Player p1 = new Player("Neymar", "Brazil",5);
      System.out.println(p1.player_detail());
      System.out.println("===================");
      Player.info();
      System.out.println("2------------------");
      Player p2 = new Player("Ronaldo", "Portugal", 7);
      System.out.println(p2.player_detail());
      System.out.println("===================");
      Player.info();
      System.out.println("3------------------");
      Player p3 = new Player("Messi", "Argentina", 6);
      System.out.println(p3.player_detail());
      System.out.println("===================");
      Player.info();
      System.out.println("4------------------");
      Player p4 = new Player("Mbappe", "France", 10);
      System.out.println(p4.player_detail());
      System.out.println("===================");
      Player.info();
    }
  }
\end{lstlisting}

\clearpage
\section*{TASK2}

\subsection*{Travel.java}
\begin{lstlisting}[style=codestyle,language=Java]
package TASK2;

public class Travel {
    static private int no_of_traveller;

    private String source;
    private String destination;
    private int time;

    public Travel(String source, String destination){
        this.source = source;
        this.destination = destination;
        no_of_traveller++;
    }

    public void setTime(int time){
        this.time = time;
    }

    public void setSource(String source){
        this.source = source;
    }

    public void setDestination(String destination){
        this.destination = destination;
    }

    public String displayTravelInfo(){  
        return "Source: " + source  + "\nDestination: " + destination + "\nTime: " + time + ":00";
    }

    public static int getCount(){
        return no_of_traveller;
    }
}
\end{lstlisting}

\subsection*{TravelTester.java}
\begin{lstlisting}[style=codestyle,language=Java]
package TASK2;

public class TravelTester {
    public static void main(String[] args) {
      System.out.println("No. of Traveller = " + Travel.getCount());
      System.out.println("1================");
        
      Travel t1 = new Travel("Dhaka", "India");
      System.out.println(t1.displayTravelInfo());
      System.out.println("2================");
   
      Travel t2 = new Travel("Kuala Lampur", "Dhaka");
      t2.setTime(23);
      System.out.println(t2.displayTravelInfo());
      System.out.println("3================");
   
      Travel t3 = new Travel("Dhaka", "New_Zealand");
      t3.setTime(15);
      t3.setDestination("Germany");
      System.out.println(t3.displayTravelInfo());
      System.out.println("4================");
   
      Travel t4 = new Travel("Dhaka", "India");
      t4.setTime(9);
      t4.setSource("Malaysia");
      t4.setDestination("Canada");
      System.out.println(t4.displayTravelInfo());
      System.out.println("5================");
   
      System.out.println("No. of Traveller = " + Travel.getCount());
     }
   }
   
\end{lstlisting}

\clearpage
\section*{TASK3}

\subsection*{Cargo.java}
\begin{lstlisting}[style=codestyle,language=Java]
package TASK3;

public class Cargo {

    public static double int_capacity = 10.0;   
    public static int id_count = 0;

    private String contents;
    private double weight;
    private boolean loaded;
    private int ID;

    Cargo(String contents, double weight){
        this.contents = contents;
        this.weight = weight;
        id_count++;
        loaded = false;
        ID = id_count;
    }

    public void load(){
        if (!loaded) { 
            if (weight <= int_capacity) { 
                loaded = true;
                int_capacity -= weight;
                System.out.println("Cargo " + ID + " loaded for transport.");
            } else {
                System.out.println("Cannot load cargo, exceeds weight capacity.");
            }
        }
    }

    public void unload(){
        if (loaded) {  
            loaded = false;
            int_capacity += weight;
            System.out.println("Cargo " + ID + " unloaded.");
        }
    }

    public void details(){
        System.out.println("Cargo ID: " + ID + ", Contents: " + contents + ", Weight: " + weight + ", Loaded: " + loaded);
    }

    public static double capacity(){ 
        return int_capacity;
    }
}
\end{lstlisting}

\subsection*{CargoTester.java}
\begin{lstlisting}[style=codestyle,language=Java]
package TASK3;

public class CargoTester {
        public static void main(String[] args) {
          System.out.println("Cargo Capacity: "+ Cargo.capacity());
          System.out.println("1====================");
          Cargo a = new Cargo("Industrial Machinery", 4.5);
          a.details();
          System.out.println("2====================");
          a.load();
          System.out.println("3====================");
          Cargo b = new Cargo("Steel Ingot", 2.7);
          b.details();
          System.out.println("4====================");
          System.out.println("Cargo Capacity: "+ Cargo.capacity());
          System.out.println("5====================");
          b.load();
          System.out.println("Carg  o Capacity: "+ Cargo.capacity());
          System.out.println("6====================");
          Cargo c = new Cargo("Tree Trunks", 3.6);
          c.load();
          System.out.println("7====================");
          c.details();
          b.details();
          System.out.println("8====================");
          Cargo d = new Cargo("Processed Goods", 1.8);
          d.load();
          System.out.println("Cargo Capacity: "+ Cargo.capacity());
          System.out.println("9====================");
          b.unload();
          System.out.println("Cargo Capacity: "+ Cargo.capacity());
          System.out.println("10====================");
          c.load();
          System.out.println("11====================");
          b.details();
          System.out.println("Cargo Capacity: "+ Cargo.capacity());
        }
      }
      

\end{lstlisting}

\end{document}
